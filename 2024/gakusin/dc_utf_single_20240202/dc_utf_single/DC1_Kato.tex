%\documentclass[11pt,a4paper,uplatex,dvipdfmx]{ujarticle} 		% for uplatex
\documentclass[11pt,a4j,dvipdfmx]{jarticle} 					% for platex
\input{pieces/form00_header} % pieces
\input{pieces/kakenhi7} % pieces
\input{pieces/form01_dcpd_header} % pieces
\input{pieces/hook3} % pieces
%#Name: dc
\input{pieces/form03_dcpd_headers} % pieces
\input{pieces/form04_dc_header} % pieces
% ===== Global definitions for the Kakenhi form ======================
% 基本情報
%
%------ 研究課題名  -------------------------------------------
\newcommand{\研究課題名}{象の卵}

%----- 研究機関名と研究代表者の氏名-----------------------
\newcommand{\研究機関名}{東京大学}
\newcommand{\研究代表者氏名}{加藤臣之輔}
\newcommand{\me}{\underline{\underline{S.~Kato}}} 
%inst_general_images.tex
\newcommand{\JSPSInstructions}{%
	\textcolor{red}{(\texttt{\textbackslash JSPSInstructions}をコメントアウトしてください。)}\\
	\includegraphicsFullWidth{subject_headers/inst_general.pdf}
}

\newcommand{\PapersInstructions}{%
	\textcolor{red}{(\texttt{\textbackslash PapersInstructions}をコメントアウトしてください。)}\\
	\includegraphicsFullWidth{subject_headers/inst_papers.pdf}
}

\newcommand{\SelfReviewInstructions}{%
	\textcolor{red}{(\texttt{\textbackslash SelfReviewInstructions}をコメントアウトしてください。)}\\
	\includegraphicsFullWidth{subject_headers/inst_self_review.pdf}
}

 % pieces
% user07_header
% ===== my favorite packages ====================================
% ここに、自分の使いたいパッケージを宣言して下さい。
\usepackage{wrapfig}
%\usepackage{amssymb}
%\usepackage{mb}
%\DeclareGraphicsRule{.tif}{png}{.png}{`convert #1 `dirname #1`/`basename #1 .tif`.png}
\usepackage{lineno}

% ===== my personal definitions ==================================
% ここに、自分のよく使う記号などを定義して下さい。
\newcommand{\klpionn}{K_L \to \pi^0 \nu \overline{\nu}}
\newcommand{\kppipnn}{K^+ \to \pi^+ \nu \overline{\nu}}

% ----- 業績リスト用 -------------
\newcommand{\paper}[6]{%
	% paper{title}{authors}{journal}{vol}{pages}{year}
	\item ``#1'', #2, #3 {\bf #4}, #5 (#6).			% お好みに合わせて変えてください。
}

\newcommand{\etal}{\textit{et al.\ }}
\newcommand{\ca}[1]{*#1}	% corresponding author;   \ca{\yukawa}  みたいにして使う
\newcommand{\invitedtalk}{招待講演}

\newcommand{\yukawa}{H.~Yukawa}					% no underline
%\newcommand{\yukawa}{\underline{\underline{H.~Yukawa}}}	% with 2 underlines
\newcommand{\tomonaga}{S.~Tomonaga}

\newcommand{\prl}{Phys.\ Rev.\ Lett.\ }		% よく使う雑誌も定義すると楽

% ===== 欄外メモ ==================
\newcommand{\memo}[1]{\marginpar{#1}}
%\renewcommand{\memo}[1]{}	% 全てのメモを表示させないようにするには、行頭の"%"を消す

%\input{../../sample/simple/contents}	% skip
\input{pieces/hook5} % pieces

\begin{document}
\input{pieces/hook7} % pieces
%#Split: 01_background  
%#PieceName: p01_background
\input{pieces/p01_background_00}
\section{研究の位置づけ}
%    <<最大 1ページ>>

%s03_background
%begin 本研究の着想に至った経緯など ====================
		風呂で巨大な温泉卵について考えていて、ふと思いついた。

	準備はしようとしている。

	唯一無二。
%end 本研究の着想に至った経緯など ====================

\input{pieces/p01_background_01}

%#Split: 02_purpose_plan  
%#PieceName: p02_purpose_plan
\input{pieces/p02_purpose_plan_00}
\section{研究目的・内容等}
%    <<最大 2ページ>>

%s02_purpose_plan_dcpd
%\JSPSInstructions		% <-- 留意事項。これは消すか、コメントアウトしてください。
%begin 研究目的と研究計画shorter ====================

	象の卵の研究計画は...

	\vspace{1cm}
	\begin{thebibliography}{99}
		\bibitem{teramura} 寺村輝夫、「ぼくは王様 - ぞうのたまごのたまごやき」.
	\end{thebibliography}
%end 研究目的と研究計画shorter ====================
\input{pieces/p02_purpose_plan_01}

%#Split: 03_rights  
%#PieceName: p03_rights
\input{pieces/p03_rights_00}
\section{人権の保護及び法令等の遵守への対応}
%    <<最大 1ページ>>

% s09_rights
%begin 人権の保護及び法令等の遵守への対応 ====================
	象の卵のES細胞の培養、象のクローンの生成などは行わない。

	\LaTeX の便利な機能については、\texttt{egg\_***.tex} や\texttt{sample\_pdf/egg\_***.pdf}をご覧ください。
%end 人権の保護及び法令等の遵守への対応 ====================

\input{pieces/p03_rights_01}

%#Split: 04_abilities  
%#PieceName: p04_abilities
\input{pieces/p04_abilities_00}
\section{研究遂行力の自己分析}
%    <<最大 2ページ>>

% s14_abilities
%\SelfReviewInstructions\\% <-- 留意事項:これは消すか、コメントアウトしてください。
\noindent
\textbf{(1) 研究に関する自身の強み}\\
%\PapersInstructions\\ %<-- 留意事項:これは消すか、コメントアウトしてください。
%begin 自己分析 ====================
	申請者はこの1年の間で様々な学会やカンファレンスに赴き足を運んだ。
	それってすごいと思う。
	\begin{enumerate}
		\paper{Search for whale eggs}{\yukawa\ \etal}{Rev.\ Oceanic Mysteries}{888}{99}{2017}
			\label{pub:whale}
		
		% 下のように書いてもいいけど、めんどくさいし、表示の仕方を変えようとしたら大変。
		%\item ``Egg of Elephant-Bird'', 
		%		\underline{A.~Cooper},
		%		Nature, {\bf 409}, 704-707 (2001).	% 	
		%\input{jack_pub}	% << only for demonstration. Please delete it or comment it out.
	\end{enumerate}
%end 自己分析 ====================

\vspace{5mm}
\noindent
\textbf{(2) 今後研究者として更なる発展のため必要と考えている要素}\\
%begin 今後必要な要素 ====================
研究費を獲得する術。
%end 今後必要な要素 ====================

\input{pieces/p04_abilities_01}

%#Split: 05_my_ambitions  
%#PieceName: p05_my_ambitions
\input{pieces/p05_my_ambitions_00}
\section{目指す研究者像等}
%    <<最大 1ページ>>

% s17_my_ambitions
\noindent
\textbf{(1)目指す研究者像 {\scriptsize ※目指す研究者像に向けて身に付けるべき資質も含め記入してください。}}

%begin 目指す研究者像 ====================
私が目指す研究者像は象である。
象は寡黙で思慮深く、自分の目指す目的地に向かって悠然と大地を歩く。
私は象になりたい。
%end 目指す研究者像 ====================

\vspace{5mm}
\noindent
\textbf{(2)上記の「目指す研究者像」に向けて、特別研究員の採用期間中に行う研究活動の位置づけ}

%begin 研究活動の位置づけ ====================
研究者として象に近づき、象の卵を探すために密着して象を観察することにより、自分を\\
「研究者としての象」に近づける。
%end 研究活動の位置づけ ====================

\input{pieces/p05_my_ambitions_01}

%#Split: 99_tail
\input{pieces/hook9} % pieces
\end{document}

